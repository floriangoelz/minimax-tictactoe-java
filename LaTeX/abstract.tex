% !TEX root =  master.tex
\chapter*{Abstract}

\begin{flushright}
    \textit{\DerAutorDerArbeit - \DieMatrikelnummer - \DieKursbezeichnung}
\end{flushright}

\textbf{\DerTitelDerArbeitEnglisch}

The goal of this paper was the implementation of the Minimax algorithm and the Tic-tac-toe game in Java,
followed by a performance comparison with the programming language Python. After a short explanation
of the motivation and goal there is an explanation of how Tic-tac-toe works and which course a game can take.
This explanation is followed by a description of the Minimax algorithm, which is used as artificial intelligence.
Based on the implementation, a human player should be able to compete against the algorithm. At this point the
most important basics are explained. The next step is showing the differences between the programming languages
Java and Python, which includes, for example, the different memory management. Based on the previously determined
differences, a theory is made, in which Java is assessed to be significantly faster than Python. In terms of
memory consumption, there were no major differences expected. Afterwards the paper provides a rough 
overview of the classes the code was implemented in. This is because the implementation should not be in focus. 
Instead, only the aspects that are important for the performance are explained in more detail.
This includes how the games state is saved as a bitmask and how caching is implemented. At the end of the paper
there is the performance test of the final implementation, which confirms the theory with regard to the execution
time and shows that Java is a lot faster than Python. However, it was not expected that the memory consumption
in Python would be that much lower than it was in Java.