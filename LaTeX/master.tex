\documentclass[
	11pt,
	BCOR=5mm,
	DIV=12,
	headinclude=on,
	footinclude=off,
	footheight = 23.39284pt,
	parskip=half,
	bibliography=totoc,
	listof=entryprefix,
	toc=listof,
	numbers=noenddot,
	plainfootsepline]{scrreprt}

%	Konfigurationsdatei einziehen

\input{config}

\begin{document}

%% BITTE GEBEN SIE HIER WICHTIGE INFORMATIONEN DER ARBEIT AN!
%% DIESE INFORMATIONEN MÜSSEN GESETZT SEIN, UM TITELBLATT, ABSTRACT UND
%% EIGENSTÄNDIGKEITSERKLÄRUNG AUTOMATISCH ANZUPASSEN!
\TitelDerArbeit{Implementierung vom Minimax Algorithmus und Tic-Tac-Toe in Java}
\TitelDerArbeitEnglisch{Implementation of the Minimax-Algorithm and Tic-Tac-Toe in Java}
\AutorDerArbeit{Fabio Jungmann \& Florian Gölz }
\Matrikelnummer{7527342, 9223147 }
\Kurs{TINF18AI2}

% !TEX root =  master.tex
\begin{titlepage}
\begin{minipage}{\textwidth}
		\vspace{-2cm}
		\noindent \hfill   \includegraphics{img/logo.jpg}
\end{minipage}
\vspace{3.2em}
\sffamily
\begin{center}
	\textsf{\textbf{HAUSARBEIT}}\\[1cm]
    \textsf{\textbf{\large{}\DerTitelDerArbeit}} \\[1cm]
	\textsf{im Studiengang}\\[3mm]
	\textsf{Angewandte Informatik} 
	\textsf{an der Dualen Hochschule Baden-Württemberg Mannheim} \\[0.7cm]
	\textsf{von}\\[5mm]
	\textsf{\DerAutorDerArbeit} \\[0.6cm]
	\textsf{11.06.2021} \\[0.8cm]
\vfill


\begin{minipage}{\textwidth}

\begin{tabbing}
	Wissenschaftlicher Betreuer: \hspace{0.6cm}\=\kill
	Matrikelnummer, Kurs \>\DieMatrikelnummer \DieKursbezeichnung \\[3mm]
	Ausbildungsfirma \>Detect Value AG, Roche Diagnostics GmbH \\[3mm] 
	Dozent \> Prof. Dr. Karl Stroetmann \\[3mm]
\end{tabbing}
\end{minipage}

\end{center}

\end{titlepage}

\pagenumbering{Roman} % Römische Seitennummerierung
\normalfont

%	Ehrenwörtliche Erklärung
% Wird die folgende Zeile auskommentiert, erscheint die ehrenwörtliche
% Erklärung im Inhaltsverzeichnis.

% \addcontentsline{toc}{chapter}{Ehrenwörtliche Erklärung}
% !TEX root =  master.tex
\clearpage
\chapter*{Ehrenwörtliche Erklärung}

Wir versichern hiermit, dass wir die vorliegende Arbeit
 mit dem Thema \newline\\[2mm] \textbf{\textit{\DerTitelDerArbeit}} \newline\\[2mm] selbstständig verfasst und keine anderen als die angegebenen Quellen und
Hilfsmittel verwendet haben.

\vspace{3cm}
Ort, Datum \hfill \DerAutorDerArbeit


% Sperrvermerk
%\input{sperrvermerk.tex}

%	Anmerkung zur Verwendung geschlechterspezifischer Bezeichnungen (Inklusionsverweis)
% !TEX root =  master.tex
\chapter*{Anmerkung}

Aus Gründen der besseren Lesbarkeit wird in dieser Arbeit für alle personenbezogenen Begriffe nur die männliche Sprachform verwendet. Sämtliche Personenbezeichnungen gelten gleichermaßen für jedes Geschlecht.

%--------------------------------
% Verzeichnisse - nicht benötige Verzeichnisse bitte auskommentieren / löschen.
%--------------------------------


%	Kurzfassung in Deutsch und Englisch
% !TEX root =  master.tex
\chapter*{Kurzfassung}

\begin{flushright}
    \textit{\DerAutorDerArbeit - \DieMatrikelnummer - \DieKursbezeichnung}
\end{flushright}

\textbf{\DerTitelDerArbeit}
\addcontentsline{toc}{chapter}{Kurzfassung}
% !TEX root =  master.tex
\chapter*{Abstract}

\begin{flushright}
    \textit{\DerAutorDerArbeit - \DieMatrikelnummer - \DieKursbezeichnung}
\end{flushright}

\textbf{\DerTitelDerArbeitEnglisch}

\addcontentsline{toc}{chapter}{Abstract}

%	Inhaltsverzeichnis
\tableofcontents

%	Abbildungsverzeichnis
%\listoffigures

%	Tabellenverzeichnis
%\listoftables

%	Listingsverzeichnis
%\lstlistoflistings

% 	Algorithmenverzeichnis
%\listofalgorithms

% 	Abkürzungsverzeichnis (siehe Datei acronyms.tex!)
%\clearpage
\chapter*{Abkürzungsverzeichnis}	
\addcontentsline{toc}{chapter}{Abkürzungsverzeichnis}

\begin{acronym}[RDBMS]
    \acro{AOSP}{Android Open Source Project}
    \acro{API}{Application Programming Interface}
    \acro{APS}{Active Pixel Sensor}
    \acro{ARGB}{Alpha Red Green Blue}
    \acro{ART}{Android Runtime}
    \acro{BPM}{Beats per Minute}
    \acro{CMOS}{Complementary metal-oxide-semiconductor}
    \acro{EKG}{Elektrokardiographie}
    \acro{FFT}{Fast Fourier-Transform}
    \acro{HAL}{Hardware Abstraction Layer}
    \acro{JPEG}{Joint Photographic Experts Group}
    \acro{JVM}{Java Virtual Machine}
    \acro{LED}{Light-emitting diode}
    \acro{LOC}{Lines of Code}
    \acro{PPG}{Photoplethysmographie}
    \acro{RGB}{Red Green Blue}
    \acro{UI}{User Interface}
\end{acronym}

%\ohead{Acronyms} % Neue Header-Definition

%--------------------------------
% Start des Textteils der Arbeit
%--------------------------------
\clearpage
\ihead{\chaptername~\thechapter} % Neue Header-Definition (inner header)
\ohead{\headmark} % Neue Header-Definition (outer header)
\pagenumbering{arabic}  % Arabische Seitenzahlen

% 	Anleitungs-Datei anleitung.tex einziehen. Auf diese Weise sollten Sie versuchen, für jedes einzelne
% Kapitel eine eigene Datei anzulegen und mittels input-Kommando einzuziehen.
% Nach dem Lesen auskommentieren
% \input{anleitung}

\definecolor{light-gray}{gray}{0.95}
\newcommand{\code}[1]{\colorbox{light-gray}{\texttt{#1}}}

%	Hauptteil der Arbeit
\chapter{Einleitung}
\section{Motivation}
In der heutigen Zeit, in der immer mehr Tätigkeiten automatisiert werden, für die früher der Einsatz mindestens eines Menschen
benötigt wurde, ist es nicht verwunderlich, dass immer mehr Algorithmen entwickelt werden. Die Aufgabe der Algorithmen ist die
Automatisierung und Optimierung von Prozessen. Auf diese Weise können Prozesse (teil-)automatisiert werden, was dazu führt, dass
diese deutlich schneller, sparsamer und konstanter laufen. Fehler werden vermieden und die Qualität der Ergebnisse steigt.
Zusätzlich zur Entwicklung von neuen Algorithmen spielt auch das Optimieren von bereits existierenden Algorithmen eine entscheidende
Rolle, um weitere Ressourcen zu sparen. Nachdem auch bei der Optimierung des Codes die Grenze erreicht ist und ein Algorithmus
nicht mehr weiter verbessert werden kann, können nur noch die Umstände, unter denen der Algorithmus ausgeführt wird,verbessert werden.
Eine Möglichkeit zur Verbesserung der Umstände wäre beispielsweise der Austausch bzw. die Verbesserung der Hardware
des ausführenden Gerätes. Eine weitere wäre die Abschaltung aller weiteren Programme, während der Algorithmus läuft. Die Motivation für
diese Arbeit ist eine weitere Möglichkeit: Der Wechsel der Programmiersprache. Jede Programmiersprache hat ihre Vor- und Nachteile
in Hinsicht auf Lesbarkeit, Komplexität, Performance und Struktur. Ist es also sinnvoll die Programmiersprache zu wechseln um die Performance
des Algorithmus zu erhöhen und gleichzeitig den Speicherverbrauch zu senken?

\section{Zielsetzung}
Ziel dieser Arbeit ist es, die oben genannte Frage zu beantworten. Zu diesem Zweck sollen im Laufe dieser Arbeit sowohl das klassische Spiel Tic-Tac-Toe,
als auch der Minimax Algorithmus in der Programmiersprache Java implementiert werden. Zunächst soll jedoch erläutert werden, wie das
Spiel Tic-Tac-Toe und der Minimax Algorithmus funktionieren. Anschließend soll eine grobe Erläuterung der Implementierung in Java erfolgen.
Nach der Implementierung soll es möglich sein, dass eine Person gegen den Minimax Algorithmus antreten kann, um festzustellen, dass dieser richtig
funktioniert. Zwecks des Performancevergleiches, wurden die genannten Komponenten bereits in Python implementiert. Vor dem Vergleichen der Performance
der beiden Programmiersprachen, wird eine Darstellung der Unterschiede von Java und Python benötigt. Dies ermöglicht das Aufstellen einer These zu
Performance und Speicherverbrauch der beiden Sprachen, welche anschließend durch Testen der Programme bestätigt oder widerlegt werden kann. Um ein 
möglichst verlässliches Ergebnis zu erzielen, soll der Algorithmus sowohl in Java als auch in Python mehrmals auf dem gleichen Computer ausgeführt werden. 
Während der Ausführung werden die Zeit und der Speicherverbrauch gemessen und protokolliert. Mithilfe dieser Werte ist es anschließend möglich, einen 
Vergleich der beiden Programmiersprachen zu machen und die obige Frage zu beantworten. Zuletzt soll gezeigt werden, inwiefern sich die Implementierung des
Algorithmus weiter verbessern lässt und welche Auswirkungen die in dieser Arbeit gemachten Erkenntnisse haben könnten.
\chapter{Theorie}
\section{Das Spiel Tic-Tac-Toe}

Tic-Tac-Toe ist ein sehr altes, klassisches Strategiespiel, welches von zwei Personen gespielt wird. Eine Person
spielt hierbei das Symbol Kreuz und eine andere Person das Symbol Kreis. Das Spielfeld von Tic-Tac-Toe besteht aus
neun Feldern, in denen die entsprechenden Symbole verteilt werden können. Jeder Spieler setzt hierzu abwechselnd sein
Symbol in eines der neun Felder. Es ist nicht möglich ein bestehendes Symbol zu überschreiben oder mehrere Symbole in
ein Feld zu zeichnen. Der Anfang einer Runde Tic-Tac-Toe kann beispielsweise wie folgt aussehen:
\begin{figure}[H]
    \centering
    \includegraphics[scale=0.25]{img/tictactoe_start.png}
    \caption[Möglicher Anfang eines Tic-Tac-Toe Spiels]{Möglicher Anfang eines Tic-Tac-Toe Spiels (eigene Anfertigung)}
\end{figure}
Im ersten Teil des Bildes erkennt man die neun leeren Felder des Spielfeldes. Im zweiten Teil des Bildes fängt der erste
Spieler damit an, sein erstes Kreuz in die Mitte des Spielfeldes zu setzen. Der zweite Spieler ist nun am Zug und setzt
im dritten Teil des Bildes seinen Kreis in die obere rechte Ecke des Spielfeldes. Als nächstes wäre Spieler eins wieder an
der Reihe und dürfte sein nächstes Kreuz setzen. Ziel des Spiels ist es, drei gleiche Symbole in einer Reihe, Spalte oder
Diagonale zu haben. Die folgende Abbildung soll dieses Verfahren nochmal genauer erläutern:
\begin{figure}[H]
    \centering
    \includegraphics[scale=0.25]{img/tictactoe_endings.png} 
    \caption[Mögliches Ende eines Tic-Tac-Toe Spiels]{Mögliches Ende eines Tic-Tac-Toe Spiels (eigene Anfertigung)}
\end{figure}
Im ersten Teil der Abbildung ist zu sehen, wie der Spieler mit dem Kreuzsymbol drei Kreuze auf einer Diagonale unterbringen
konnte. In diesem Fall ist die Runde beendet und dieser Spieler hat die Runde gewonnen. Ob das Spiel über drei Symbole in einer
Reihe, Spalte oder Diagonale gewonnen wird, ist nicht relevant. Ebenfalls spielt es keine Rolle, in welcher Reihenfolge die 
Symbole gesetzt wurden. Im zweiten Teil des Bildes zeigt sich ein weiteres mögliches Ende für eine Runde Tic-Tac-Toe. Bei diesem
Ende ist das komplette Spielfeld ausgefüllt und es ergeben sich keine drei Symbole in einer Reihe, Spalte oder Diagonale. 
Entsprechend ist die Runde unentschieden ausgegangen.

\section{Der Minimax-Algorithmus}
Nachdem das Spiel Tic-Tac-Toe erklärt wurde, soll nun mit Hilfe des Minimax Algorithmus eine künstliche Intelligenz 
entwickelt werden, die in einem Spiel gegen einen menschlichen Spieler immer einen optimalen Zug ausführt. Der Minimax 
Algorithmus baut dabei auf der Funktion \code{value} auf, die wie folgt definiert ist:

\[value: States \times Players \rightarrow \{-1,0,1\}\]

Die Rückgabewert symbolisieren dabei den bestmöglichen Ausgang eines Zuges:

\begin{itemize}
    \item -1 bedeutet, dass der Spieler keine Möglichkeit mehr hat eine Niederlage zu verhindern
    \item 0 bedeutet, dass im besten Fall ein Unentschieden erzielt werden kann
    \item 1 bedeutet, dass ein Sieg möglich ist
\end{itemize}

Die Funktion bekommt also einen Zustand des Spielbretts und den Spieler für den der Wert berechnet werden soll.
Um den Rückgabewert zu berechnen, werden sämtliche mögliche Spielverläufe berechnet und 

\chapter{Praxis}

\section{Implementierung von Tic-Tac-Toe und Minimax}
Ziel dieses Kapitels ist es die Implementierung in Java zu erläutern. Da der Fokus dieser Arbeit jedoch auf der Performance
des Algorithmus in Java liegt und eine detaillierte Erläutertung der Implementierung sehr aufwendig wäre, folgt lediglich
ein grober Überblick über die meisten Aspekte der Implementierung. Lediglich Teile der Implementierung, die besonders relevant
für die Performance des Programms sind werden detaillierter erläutert. 

\subsection{Objektorientierung: Unterteilung in Klassen}
Zunächst folgt eine grobe Erläuterung der drei Klassen, in die die Implementierung unterteilt wurde.
Diese Klassen sind auch im folgenden Klassendiagramm zu sehen:
\begin{figure}[H]
    \centering
    \includegraphics[scale=0.3]{img/uml_diagram.png}
    \caption[Vollständiges Klassendiagramm der Implementierung]{Vollständiges Klassendiagramm der Implementierung (eigene Anfertigung)} % TODO besseres Bild, sobald es richtig ist
    \label{fig:uml}
\end{figure}

\textbf{Main.java:} Diese Klasse enthält die main-Methode und ist der Startpunkt für das Programm. Sie ist sie dafür
verantwortlich, dass Objekte der anderen beiden Klassen erstellt und verwaltet werden. Die gesamte Logik, welche es ermöglicht, dass 
eine Person eine Runde Tic-Tac-Toe gegen den Minimax-Algorithmus spielen kann, ist in dieser Klasse. Hierzu gehören beispielsweise 
das Abfragen des nächsten Zuges der Person und des Algorithmus sowie alle Ein- und Ausgaben, die zur Bedienung des Programmes notwendig sind. 
Außerdem enthält diese Klasse die Funktion zur Messung der Performance, welche in Kapitel \ref{chap:Performancevergleich} zum Einsatz kommt.

\textbf{TicTacToe.java:} Die Klasse TicTacToe ist, wie der Name schon sagt, die Klasse, in der die gesamte benötigte Logik für eine Runde
Tic-Tac-Toe enthalten ist. In dieser Klasse sind sowohl die möglichen Zustände, bei denen ein Spieler das Spiel gewonnen hat als auch der
aktuelle Zustand des Spielbretts gespeichert. Ebenfalls wird der Status des Spiels (siehe enum in Abbildung \ref{fig:uml}), sowie der Spieler, der aktuell
am Zug ist, gespeichert. Des Weiteren sind alle Funktionen enthalten, die zum Anzeigen oder Verändern des Spielfeldes benötigt werden. Hierzu
gehört auch eine Funktion, die ermittelt, ob der gegebene Zug zulässig ist oder das gewünschte Feld bereits belegt ist.

\textbf{Minimax.java:} Innerhalb der Klasse Minimax ist die gesamte Logik für den Minimax Algorithmus enthalten. Hierzu gehören alle Funktionen,
welche benötigt werden, um den bestmöglichen Spielzug bei einem bestimmten Spielstand zu ermitteln. Um alle notwendigen Parameter zu erhalten,
die für die Ermittlung des nächsten Spielzuges benötigt werden, ruft die Klasse Minimax teilweise statische Funktionen der Klasse TicTacToe auf,
welche die notwendigen Werte zurückliefern. Die relevanteste Funktion in dieser Klasse ist die rekursive Funktion \code{value(int state, int player)}, 
da diese in Kapitel \ref{chap:performancevergleich} zur Messung der Performance aufgerufen wird. Die Funktion dient der Bewertung eines möglichen 
Spielzustands anhand der möglichen Gewinnmöglichkeiten (siehe auch Kapitel \ref{chap:Minimax}).

\subsection{Performancerelevante Implementierungen im Detail}
\label{chap:implementierungen}
\subsubsection{Spielstand als Bitmaske}

\subsubsection{Caching von Folgezuständen}

\section{Performancevergleich von Java und Python}
\label{chap:performancevergleich}

Nachdem der Minimax-Algorithmus für das Spiel Tic-Tac-Toe in Java implementiert wurde, soll nun die 
Performance mit der Umsetzung in Python verglichen werden. Hierfür wird bei beiden Versionen die Funktion 
\code{value} für ein leeres Spielfeld mit Spieler 0 aufgerufen. D. h. die Funktion \code{value} berechnet 
alle möglichen Spielzustände und deren Wert. Die CPU-Zeiten beziehen sich auf eine Testreihe mit einem 
Intel i7-9700K. Für jeden Test werden jeweils fünf Durchgänge mit und ohne Memoisierung durchgeführt 
und daraus der Durchschnittswert berechnet.

\begin{table}[H]
    \centering
    \begin{tabular}{|l|ll|l|}
        \hline
        \multicolumn{4}{|l|}{\textbf{RAM}}                                             \\ \hline
                                & ohne memoize & mit memoize &                         \\ \hline
        \multirow{1}{*}{Python} &              & 808         & \multirow{7}{*}{in KB}  \\ \cline{1-3}
        \multirow{5}{*}{Java}   & 12875        & 1515        &                         \\
                                & 13392        & 1515        &                         \\
                                & 13387        & 1515        &                         \\
                                & 12881        & 1515        &                         \\
                                & 13391        & 1514        &                         \\ \cline{1-3}
        average                 & 13185,2      & 1514,8      &                         \\ \hline
    \end{tabular}
    \caption{Ergebnisse der Performancetestreihe für den genutzten Arbeitsspeicher}
\end{table}

Da für den belegten Arbeitsspeicher der Python Implementierung keine verlässlichen Messwerte gemessen werden 
konnten, wird der Messwert aus dem Jupyter-Notebook (Bitboard mit Memoisierung) verwendet. Auffallend ist, 
dass die Java Implementierung selbst mit Memoisierung, fast doppelt so viel Speicher verbraucht wird. 
Darüber hinaus kann durch Memoisierung bei der Implementierung in Java der Speicherverbrauch um den 
Faktor 8,5 reduziert werden.

\begin{table}[H]
    \centering
    \begin{tabular}{|l|ll|l|}
        \hline
        \multicolumn{4}{|l|}{\textbf{CPU Time}}                                        \\ \hline
                                & ohne memoize & mit memoize &                         \\ \hline
        \multirow{5}{*}{Python} & 2520         & 42,6        & \multirow{12}{*}{in ms} \\
                                & 2520         & 45          &                         \\
                                & 2520         & 44          &                         \\
                                & 2510         & 42          &                         \\
                                & 2510         & 44          &                         \\ \cline{1-3}
        average                 & 2516         & 43,92       &                         \\ \cline{1-3}
        \multirow{5}{*}{Java}   & 43           & 8           &                         \\
                                & 43           & 8           &                         \\
                                & 44           & 9           &                         \\
                                & 45           & 7           &                         \\
                                & 42           & 8           &                         \\ \cline{1-3}
        average                 & 43,4         & 8           &                         \\ \hline
    \end{tabular}
    \caption{Ergebnisse der Performancetestreihe für die benötigte Zeit}
\end{table}

Die Schnelligkeit der Python-Implementierung mit Memoisierung ist in etwa gleich zur Java-Implementierung 
ohne Memoisierung. Mit Memoisierung kann die Laufzeit durch die Verwendung von Java nochmal um 80\% verringert 
werden.

\section{Java vs. Python}
Neben den in Kapitel \ref{chap:implementierungen} Unterschieden bei der Implementierung, können die in Kapitel 
\ref{chap:performancevergleich} ermittelten Performanceunterschiede durch Unterschiede in der Funktionsweise der 
Sprachen erklärt werden.

\subsection{Memory Management}
Während Python oft als unperformant wahrgenommen wird, konnte Python beim Arbeitsspeicherverbrauch im Vergleich 
zu Java punkten. Dies kann vorallem durch unterschiede beim Speichermanagement der beiden Sprachen erklärt werden. 

Arbeitsspeicher wird in Python über einen Heap, der intern verwaltet wird. Wird also Arbeitsspeicher benötigt, 
wird dieser durch das Memory-Management vom Betriebssystem angefordert. Objekte bleiben nur so lange im Speicher 
wie sie benötigt werden. Python nutzt das sogenannte Reference Counting um Objekte zu speichern.

\begin{lstlisting}[caption={Codebeispiel zur Verwaltung von Objekten in Python}]
    a = 9
    b = 9
    c = 11
\end{lstlisting}

In diesem Beispiel wird der Wert 9 als Objekt im Speicher erstellt und Variable \code{a} zugewiesen. Da 
Das Objekt mit dem Wert 9 schon existiert, wir Variable {b} als weitere Referenz hinzugefügt. Für Variabl 
\code{c} heißt das, dass ein Objekt mit dem Wert 11 angelegt werden muss und \code{c} als Referenz hinterlegt wird. 
Beträgt die Anzahl der Referenzen eines Objekts Null, so wird es aus den Speicher entfernt. Die Referenzen 
werden zyklisch vom Garbage Collector überprüft und Objekte ohne Referenzen entfernt. Reference Counting 
hat den Vorteil, dass nicht genutzter Speicher sehr schnell erkannt wird. Allerdings entsteht durch das Zählen 
der Referenzen ein gewisser Overhead, der im Vergleich zu anderen Garbage Collecting Verfahren aber eher gering ausfällt.

%TODO: Garbage Collection Java
\chapter{Fazit \& Ausblick}

\section{Fazit}

\section{Ausblick}
Kleine Liste was vorkommen könnte (kannst aber selbstverständlich auch selbst irgendwas finden :D) 

- Speicherung in 1x Byte und 1x Short (8 Bit aus dem Byte und 16 Bit aus dem Short), um ein weiteres Byte zu sparen (int hat 4 byte). Evtl. auch
Speicherung mit 18 Booleans (und ggf. in einem zweidimensionalen array vom typ boolean), aber ka ob das schneller ist. Als These könntest dus aber bringen

- Weitere Optimierung (auch in Einleitung angedeutet): Weiteres Caching (z. B. die NextStates und nicht nur deren Wert), Wechsel der Programmiersprache, Upgrade der CPU - 
oder sogar einmalige Speicherung des optimalen Zuges bei jedem Zustand, sodass dieser künftig direkt ausgewählt werden kann und der Algorithmus nur einmalig
durchlaufen werden muss, um alle Werte zu errechnen und zu Speichern

%	Literaturverzeichnis
\clearpage
\ihead{\chaptername~\thechapter} 
\ohead{\headmark} 
\pagenumbering{Roman}
\setcounter{page}{9}

\printbibliography[title=Literaturverzeichnis]

\cleardoublepage

%	Glossar
%\addcontentsline{toc}{chapter}{Glossar}
%% !TEX root =  master.tex
\chapter*{Glossar}



%\newpage

% Der Anhang beginnt hier - jedes Kapitel wird alphabetisch aufgezählt. (Anhang A, B usw.)
\appendix
\ihead{\appendixname~\thechapter} % Neue Header-Definition

%appendix.tex einziehen
\addcontentsline{toc}{chapter}{Anhang}
%\input{appendix}

\end{document}