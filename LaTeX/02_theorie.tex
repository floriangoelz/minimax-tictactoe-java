\chapter{Theorie}
\section{Der Minimax-Algorithmus}

Der Minimax-Algorithmus, ist ein Algorithmus, der für ein gegebenes Spiel eine Spielstrategie ermittelt. 
Das Spiel muss dabei ein Zwei-Personen-Nullsummenspiel mit perfekter Information sein. Die Idee ist, mögliche 
Spielverläufe zu bewerten und damit für einen gegebenen Spielzustand den Zug zu wählen, der zum besten Spielzustand 
für den zugewiesenen Spieler führt. Als Basis für diese Entscheidung dient eine Funktion utility. Diese Funktion 
nimmt den aktuellen Spielstand und überprüft, in welchem Zustand sich die übergebene Spielsituation befindet. 

\begin{itemize}
    \item Kann im besten Fall ein Sieg erzielt werden, wird der Wert 1 zurückgegeben
    \item Ist nur ein Unentschieden möglich, wird der Wert 0 zurückgegeben
    \item Ist eine Niederlage nicht zu abzuwenden, wird der Wert -1 zurückgegeben
\end{itemize}

Um dies zu ermitteln, wird ein Baum mit allen möglichen Spielverläufen berechnet. Die utility Funktion überprüft 
nun rekursiv für jeden potentiellen Spielzustand, ob das Spiel beendet wurde. Ist dies der Fall, ist das 
Ende des Baumes erreicht und 