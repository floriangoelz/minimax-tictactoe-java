\chapter{Einleitung}
\section{Motivation}
In der heutigen Zeit, in der immer mehr Tätigkeiten automatisiert werden, für die früher der Einsatz mindestens eines Menschen
benötigt wurde, ist es nicht verwunderlich, dass immer mehr Algorithmen entwickelt werden. Die Aufgabe der Algorithmen ist die
Automatisierung und Optimierung von Prozessen. Auf diese Weise können Prozesse (teil-)automatisiert werden, was dazu führt, dass
diese deutlich schneller, sparsamer und konstanter laufen. Fehler werden vermieden und die Qualität der Ergebnisse steigt.
Zusätzlich zur Entwicklung von neuen Algorithmen spielt auch das Optimieren von bereits existierenden Algorithmen eine entscheidende
Rolle, um weitere Ressourcen zu sparen. Nachdem auch bei der Optimierung des Codes die Grenze erreicht ist und ein Algorithmus
nicht mehr weiter verbessert werden kann, können nur noch die Umstände, unter denen der Algorithmus ausgeführt wird,verbessert werden.
Eine Möglichkeit zur Verbesserung der Umstände wäre beispielsweise der Austausch bzw. die Verbesserung der Hardware
des ausführenden Gerätes. Eine weitere wäre die Abschaltung aller weiteren Programme, während der Algorithmus läuft. Die Motivation für
diese Arbeit ist eine weitere Möglichkeit: Der Wechsel der Programmiersprache. Jede Programmiersprache hat ihre Vor- und Nachteile
in Hinsicht auf Lesbarkeit, Komplexität, Performance und Struktur. Ist es also sinnvoll die Programmiersprache zu wechseln um die Performance
des Algorithmus zu erhöhen und gleichzeitig den Speicherverbrauch zu senken?

\section{Zielsetzung}
Ziel dieser Arbeit ist es, die oben genannte Frage zu beantworten. Zu diesem Zweck sollen im Laufe dieser Arbeit sowohl das klassische Spiel Tic-Tac-Toe,
als auch der Minimax Algorithmus in der Programmiersprache Java implementiert werden. Zunächst soll jedoch erläutert werden, wie das
Spiel Tic-Tac-Toe und der Minimax Algorithmus funktionieren. Anschließend soll eine grobe Erläuterung der Implementierung in Java erfolgen.
Nach der Implementierung soll es möglich sein, dass eine Person gegen den Minimax Algorithmus antreten kann, um festzustellen, dass dieser richtig
funktioniert. Zwecks des Performancevergleiches, wurden die genannten Komponenten bereits in Python implementiert. Vor dem Vergleichen der Performance
der beiden Programmiersprachen, wird eine Darstellung der Unterschiede von Java und Python benötigt. Dies ermöglicht das Aufstellen einer These zu
Performance und Speicherverbrauch der beiden Sprachen, welche anschließend durch Testen der Programme bestätigt oder widerlegt werden kann. Um ein 
möglichst verlässliches Ergebnis zu erzielen, soll der Algorithmus sowohl in Java als auch in Python mehrmals auf dem gleichen Computer ausgeführt werden. 
Während der Ausführung werden die Zeit und der Speicherverbrauch gemessen und protokolliert. Mithilfe dieser Werte ist es anschließend möglich, einen 
Vergleich der beiden Programmiersprachen zu machen und die obige Frage zu beantworten. Zuletzt soll gezeigt werden, inwiefern sich die Implementierung des
Algorithmus weiter verbessern lässt und welche Auswirkungen die in dieser Arbeit gemachten Erkenntnisse haben könnten.