\chapter{Fazit \& Ausblick}

\section{Fazit}
Ziel dieser Arbeit war es den Minimax-Algorithmus am Beispiel des Spiels Tic-Tac-Toe in Java zu implementieren. Auf dieser Basis 
sollten dann die Performanceunterschiede zwischen Java und Python getestet und analysiert, sowie Unterschiede bei der Implementierung 
aufgezeigt werden. Da Python keinen JIT-Compiler verwendet und keine statische Typisierung gewährleistet werden kann, wurde die These 
aufgestellt, dass die Java-Implementierung beim Performancetest besser abschneiden wird als die Python-Umsetzung. Bei der Implementierung 
in Java ist besonders die deutlich höhere Anzahl an Codezeilen und die schlechtere Lesbarkeit im Vergleich zu Python aufgefallen. 
Unklar waren hingegen die Unterschiede bei der Nutzung des Arbeitsspeichers, da beide Sprachen hierbei Vor- und Nachteile vorweisen. 
Es konnte gezeigt werden, dass beim Aufruf der \code{value} Funktion die Laufzeit durch die Nutzung von Java und Memoisierung um etwa 
80\% gesenkt werden kann. Überraschend war hingegen, dass bei der Implementierung in Java trotz Memoisierung fast doppelt so viel 
Arbeitsspeicher belegt wurde, als bei der Python Implementierung.


\section{Ausblick}
Es konnte gezeigt werden, dass die Wahl der Programmiersprache einen großen Einfluss auf die Performance eines Programms hat. 
Hardwarenahe, kompilierte Sprachen haben den Vorteil, dass weniger Schritte beim Übersetzungsprozess benötigt werden und während der Laufzeit 
keine Übersetzung mehr stattfindet. Möchte man also die Performance weiter verbessern, könnte eine Umsetzung in der Sprache C bzw. C++ sinnvoll 
sein. Wobei dabei die Plattformunabhängigkeit verloren geht. 

Die Nutzung von Memosierung zum Speichern von bereits berechneten Ergebnissen der \code{value} Funktion hat sich als sinnvolle 
Optimierung erwiesen. Dieser Ansatz kann auch auf weitere Funktionen angewandt werden, bei denen redundante Funktionsaufrufe vorkommen. 
Ein Kandidat wäre hierbei die \code{nextStates} Funktion, die für einen gegebenen Zustand und Spieler alle möglichen Folgezustände berechnet. 
Diese wird in der rekursiven \code{value} Funktion aufgerufen, wodurch redundante Funktionsaufrufe entstehen, die verhindert werden können. 

Bei der Performanceoptimierung muss allerdings auch immer der Anwendungsfall bedacht werden. So macht eine KI für ein Spiel wie Tic-Tac-Toe 
kaum Sinn Performanceoptimierung im Millisekundenbereich vorzunehmen. Ob der Spieler nun 45 ms oder 8 ms warten muss, ist für einen menschlichen 
Gegenspieler kaum relevant, da er den Unterschied nicht bemerkt. Bei komplexeren Spielen bzw. Algorithmen kann eine Umsetzung in einer 
geeigneteren Sprache hingegen sinnvoll sein.