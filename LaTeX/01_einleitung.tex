\chapter{Einleitung}
In der heutigen Zeit, in der immer mehr Tätigkeiten automatisiert werden, für die früher der Einsatz eines Menschen benötigt wurde,
ist es nicht verwunderlich, dass immer mehr Algorithmen entwickelt werden. Zusätzlich zur Entwicklung von neuen Algorithmen, spielt
auch das optimieren von bereits vorhandenen Algorithmen eine Rolle. Nachdem auch bei der Optimierung die Grenze erreicht ist und ein
Algorithmus nicht mehr weiter verbessert werden kann, können nur noch die Umstände, unter denen der Algorithmus ausgeführt wird,
verbessert werden. So kann beispielsweise die Optimierung und der Austausch der Hardware oder der Wechsel auf eine andere
Programmiersprache das Ergebnis noch weiter verbessern. Diese Arbeit beschäftigt sich mit dem Minimax-Algorithmus und dem Spiel
Tic-Tac-Toe, wobei beides in der Programmiersprache Java umgesetzt wird. Im späteren Verlauf der Arbeit findet außerdem ein
Performancevergleich zwischen Java und Python statt.