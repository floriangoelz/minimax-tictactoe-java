% !TEX root =  master.tex
\chapter*{Kurzfassung}

\begin{flushright}
    \textit{\DerAutorDerArbeit - \DieMatrikelnummer - \DieKursbezeichnung}
\end{flushright}

\textbf{\DerTitelDerArbeit}

Ziel der Arbeit war die Implementierung vom Minimax Algorithmus und vom Spiel Tic-Tac-Toe in Java,
gefolgt von einem Performancevergleich mit der Programmiersprache Python. Nach einer kurzen Erläuterung
der Motivation und des Ziels der Arbeit wird zunächst erläutert, wie das Spiel Tic-Tac-Toe
aufgebaut ist und welchen Spielverlauf ein Spiel annehmen kann. Dieser Erläuterung folgt die
Beschreibung des Minimax Algorithmus, welcher als künstliche Intelligenz zum Einsatz kommt.
Auf Basis der Implementierung sollte ein menschlicher Spieler in der Lage sein, gegen den
Algorithmus zu spielen. An dieser Stelle sind die wichtigsten Grundlagen erläutert und es werden 
zunächst die Unterschiede zwischen den Programmiersprachen Java und Python dargestellt. Hierzu gehört
beispielsweise die unterschiedliche Speicherverwaltung. Auf Basis der zuvor ermittelten Unterschiede
ergibt sich eine These zum Performanceunterschied der beiden Sprachen, bei dem Java als deutlich 
schneller als Python eingeschätzt wurde. Hinsichtlich des Arbeitsspeicherverbrauchs wurden keine großen Unterschiede erwartet.  
Es folgt die Implementierung in Java. Da diese nicht im Fokus der Arbeit stehen soll, wird hierbei nur ein grober Überblick über die 
Klassen gegeben. Lediglich die Speicherung des Spielzustandes als Bitmaskierung sowie das Caching werden
genauer erklärt, da diese wichtig für die Performance der Implementierung sind. Zuletzt folgt ein
Performancetest, welcher die These hinsichtlich der Laufzeit bestätigt und zeigt, dass Java schneller ist als
Python. Nicht erwartet wurde hierbei allerdings, dass der Speicherverbrauch in Python so viel geringer 
ist als in Java.